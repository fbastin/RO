\chapter{Programmation dynamique}

Supposons un problème décomposé en étapes.
Chaque étape possède un certain nombre d’états correspondant aux conditions possibles au début d’une étape.
A chaque étape, nous devons prendre une décision qui associe à l’état courant l’état au début de la prochaine étape.
Dans le cas d’un problème d’optimisation, la programmation dynamique identifie une politique optimale: une décision optimale à chaque étape pour chaque état possible.

\section{Principe d’optimalité de Bellman}

Étant donné un état courant, la politique optimale pour les étapes à venir est indépendante des
décisions prises aux étapes précédentes.
Lorsque ce principe est satisfait, nous pouvons recourir à l’approche suivante pour résoudre un problème ayant $N$ étapes:
\begin{itemize}
\item
identifier une décision optimale pour la dernière étape ($N$);
\item
pour chaque étape $n$, de $N-1$ à 1: étant donné une politique
optimale pour l’étape $n+1$, identifier une politique optimale
pour l’étape $n$ au moyen d’une relation de récurrence.
\end{itemize}
Lorsque n = 1, nous obtenons une politique optimale pour le problème.

Nous utiliserons les notations suivantes:
\begin{itemize}
\item
 $N$: nombre d’étapes;
\item
 $n$: indice de l’étape courante;
\item
 $s_n$: état au début de l’étape $n$;
\item
 $x_n$: variable de décision associée à l’étape $n$;
\item
 $f_n(s_n,x_n)$: contribution des étapes $n$, $n+1$,\ldots, $N$ à la valeur de l’objectif lorsqu’on se trouve à l’état $s_n$ au début de l’étape $n$, que la décision est $x_n$, et que des décisions optimales sont prises par la suite;
\item
$x_n^*$: solution optimale à l’étape $n$;
\item
 $f_n^*(s_n) = f_n(s_n,x_n^*) = \min \lbrace f_n(s_n,x_n) \rbrace$ ou $\max \lbrace f_n(s_n, x_n) \rbrace$.
\end{itemize}

A chaque itération de la méthode, on construira un tableau représentant
\begin{itemize}
\item
la liste des états possibles à l’étape $n$;
\item
les valeurs possibles de la variable $x_n$;
\item
la contribution à la valeur de l’objectif $f_n(s_n,x_n)$;
\item
la valeur optimale $f^*_n(s_n)$ et la solution optimale $x_n*$.
\end{itemize}

\begin{tabular}{|c|c|c|c|c|c|}
\hline
$s_n$ & $x_n = a$ & $x_n = b$ & $x_n = c$ & $f_n^*(s_n)$ & $x_n^*$ \\
\hline
1 & $f_n(1,a)$ & $f_n(1,b)$ & $f_n(1,c)$ & $f_n^*(1)$ & \\
\hline
2 & $f_n(2,a)$ & $f_n(2,b)$ & $f_n(2,c)$ & $f_n^*(2)$ & \\
\hline
\end{tabular}

\begin{example}[Production-inventaire]
Une compagnie doit fournir à son meilleur client trois unités du produit $P$ durant chacune des trois prochaines semaines. Les coûts de production sont donnés dans la Table~\ref{tab:dynamic_costs}.
\begin{table}
\begin{tabular}{|c|c|c|c|}
\hline
Semaine & Production max, & Production max, & Coût unitaire, \\
& temps régulier & temps supp. & temps régulier \\
\hline
1& 2 & 2 & 300\$ \\
\hline
2 & 3 & 2 & 500\$ \\
\hline
3 & 1 & 2 & 400\$ \\
\hline
\caption{Coûts de production}
\label{tab:dynamic_costs}
\end{table}
Le coût pour chaque unité produite en temps supplémentaire est 100\$ de plus que le coût par
unité produite en temps régulier.
Le coût unitaire d’inventaire est de 50\$ par semaine
Au début de la première semaine, il y a 2 unités du produit dans l’inventaire.
La compagnie ne veut plus rien avoir dans son inventaire au bout des trois semaines.
Combien doit-on produire d’unités à chaque semaine afin de rencontrer la demande du client, tout en minimisant les coûts?

Une étape correspond ici à une semaine, $N = 3$, $n = 1, 2, 3$, et
\begin{itemize}
\item
 $s_n$, l'état au début de la semaine $n$, est le nombre d'unités de produit dans l’inventaire;
\item
 $x_n$: nombre d’unités produites lors de la semaine $n$;
\item
 $s_{n+1} = s_n + x_n – 3$ (puisque nous devons livrer 3 unités au client à chaque semaine);
\item
 $s_1 = 2$ (puisqu’il y a 2 unités au début).
\end{itemize}
Soient $c_n$, le coût unitaire de production au cours de la semaine $n$, $r_n$ la production maximale en temps régulier pendant la semaine $n$, et $m_n$ la production maximale durant la semaine $n$.
Le coût au cours de la semaine $n$ est
\[
 p_n(s_n,x_n) = c_nx_n+100\max(0,x_n–r_n)+50\max(0,s_n+x_n-3).
\]
Le coût total vaut dès lors
\[
 f_n(s_n,x_n) = p_n(s_n,x_n) + f_{n+1}^*(s_{n+1}),
\]
et le coût optimal répond à la récurrence
\[
 f_n^*(s_n) = \min \lbrace p_n (s_n, x_n) + f_{n+1}^*(s_{n+1}) \, | \, 3-s_n \rightarrow x_n \rightarrow m_n \rbrace.
\]
Si $n = 3$, nous devons poser
\[
f_4^*(s_4) = 0.
\]
De plus,
\[
s_4 = 0 = s_3 + x_3 - 3,
\]
aussi
\[
x_3 = 3 - s_3.
\]
Calculons d’abord les valeurs $f_3^*(s_3)$ et $x_3^*$.
Nous obtenons le tableau
\begin{tabular}{|c|c|c|c|}
\hline
$s_3$ & $f_3(3-s_3,s_3)$ & $f_3^*(s_3)$ & $x_3^*$ \\
\hline
0 & 1400 & 1400 & 3 \\
\hline
1 & 900 & 900 & 2 \\
\hline
2 & 400 & 400 & 1 \\
\hline
$\geq 3$ & 0 & 0 & 0 \\
\hline
\end{tabular}

Voyons maintenant comment nous pouvons calculer les valeurs $f_2^*(s_2)$ et $x_2^*$, lorsque $s_2 = 0$.
Nous devons avoir $x_2 \geq 3$ (car nous devons livrer au moins 3 unités du produit).
D'autre part,
\begin{align*}
 f_2(0,3) &= p_2(0,3) + f3*(0) = 1500 + 1400 = 2900, \\
 f_2(0,4) &= p_2(0,4) + f3*(1) = 2150 + 900 = 3050, \\
 f_2(0,5) &= p_2(0,5) + f3*(2) = 2800 + 400 = 3200, \\
 f_2^*(0) &= min{f_2(0,3), f_2(0,4), f_2(0,5)} = f_2(0,3).
\end{align*}
Par conséquent, $x_2^* = 3$.
Nous procédons de la même manière pour $s_2 = 1,2,3$.
\begin{tabular}{|c|c|c|c|c|c|c|c|c|}
\hline
$s_2$ & $x_2 = 0$ & $x_2 = 1$ & $x_2 = 2$ & $x_2 = 3$ & $x_2 = 4$ & $x_2 = 5$ & $f_2^*(s_2)$ & $x_2^*$ \\
\hline
0 & - & - & - & 2900 & 3050 & 3200 & 2900 & 3 \\
\hline
1 & - & - & 2400 & 2450 & 2600 & 2850 & 2400 & 2 \\
\hline
2 & - & 1900 & 1950 & 2000 & 2250 & 2900 & 1900 & 1 \\
\hline
$\geq 3$ & 1400 & 1450 & 1500 & 1650 & 2300 & 2950 & 1400 & 0 \\
\hline
\end{tabular}
Pour la première étape ($n = 1$), nous avons $s_1 = 2$ (il y a 2 unités au départ dans l’inventaire).
Nous devons donc avoir $x_1 \geq 1$. De plus,
\begin{align*}
 f_1(2,1) &= p_1(2,1) + f_2^*(0) = 300 + 2900 = 3200, \\
 f_1(2,2) &= p_1(2,2) + f_2^*(1) = 650 + 2400 = 3050, \\
 f_1(2,3) &= p_1(2,3) + f_2^*(2) = 1100 + 1900 = 3000, \\
 f_1(2,4) &= p_1(2,4) + f_2^*(3) = 1550 + 1400 = 2950.
\end{align*}
Par conséquent, $f_1^*(2) = f_1(2,4)$ et $x_1* = 4$.
Sous forme de tableau, cela donne:
\begin{tabular}{|c|c|c|c|c|c|c|}
\hline
$s_1$ & $x_1=1$ & $x_1=1$ & $x_1=1$ & $x_1=1$ & $f_1^*(x_1)$ & x_1^* \\
\hline
2 & 3200 & 3050 & 3000 & 2950 & 2950 & 4 \\
\hline
\end{tabular}
La politique optimale est donc:
\[
x_1* = 4,\ x_2^* = 0,\ x_3^* = 3
\]
donnant lieu à $s_2 = 3$, $s_3 = 0$.
Le coût total est $f_1^*(2) = 2950\$$.
\end{example}

\begin{example}[Affectation de ressources]

Trois équipes de recherche travaillent sur un même problème avec chacune une approche différente.
La probabilité que l’équipe $E_i$ échoue est $P[E_1] = 0.4$, $P[E_2] = 0.6$, $P[E3) = 0.8$.
Ainsi, la probabilité que les trois équipes échouent simultanément est: (0.4)(0.6)(0.8) = 0.192.
Nous voulons ajouter deux autres scientifiques afin de minimiser la probabilité d’échec.
A quelles équipes faut-il affecter ces deux scientifiques afin de minimiser la probabilité d’échec?
La Table~\ref{tab:ressources} représente les probabilités d’échec pour chaque équipe en fonction du nombre de scientifiques ajoutés.
\begin{table}[htb]
\begin{tabular}
\hline
Scientifiques ajoutés & $E_1$ & $E_2$ & $E_3$ \\
\hline
0 & 0.4 & 0.6 & 0.8 \\
\hline
1 & 0.2 & 0.4 & 0.5 \\
\hline
2 & 0.15 & 0.2 & 0.3 \\
\hline
\end{tabular}
\caption{Allocation de ressources.}
\label{tab:ressources}
\end{table}

Nous faisons correspondre l'équipe $n$ à l'étapre $n$, pour $n = 1,2,3$ ($N = 3$).
L'état $s_n$ est le nombre de scientifiques non encore affectés.
La décision $x_n$ correspond au nombre de scientifiques affectés à l’équipe $E_n$.
Soit $p_n(x_n)$ la probabilité d’échec de l’équipe $E_n$ si $x_n$ scientifiques lui sont affectés.
Alors,
\begin{align*}
 f_n(s_n,x_n) &= p_n(x_n).f_{n+1}^*(s_n-x_n) \\
 f_n^*(s_n) &= \min \lbrace f_n(s_n,x_n)\,|\,x_n=0,1,\ldots,s_n \rbrace,
\end{align*}
avec $f_4^*(0) = 1$.
Calculons d’abord $f_3^*(s_3)$ et $x_3^*$.
Nous devons avoir $s_3 = x_3$ (nous affectons tous les scientifiques non encore affectés),
de sorte que nous avons le tableau
\begin{tabular}{|c|c|c|c|}
\hline
$s_3$ & $f_3(s_3,s_3)$ & $f_3^*(s_3)$ & $x_3^*$ \\
\hline
0 & 0.8 & 0.8 & 0 \\
\hline
1 & 0.5 & 0.5 & 1 \\
\hline
2 & 0.3 & 0.3 & 2 \\
\hline
\end{tabular}
Pour calculer les valeurs $f_2^*(s_2)$ et $x_2^*$, lorsque $s_2 = 1$, observons d'abord que nous devons avoir $x_2 \leq 1$. Nous avons
\begin{align*}
 f_2(1,0) &= p_2(1,0) \times f_3^*(1) = 0.6 \times 0.5 = 0.3, \\
 f_2(1,1) &= p_2(1,1) \times f_3^*(0) = 0.4 \times 0.8 = 0.32, \\
 f_2*(1) &= \min{f_2(1,0), f_2(1,1)} = f_2(1,0).
\end{align*}
 Dès lors, $x_2^* = 0$.
Nous pouvons procéder de même pour $s_2 = 0$ et $2$.
Ceci donne le tableau ci-dessous.
\begin{tabular}
\hline
$s_2$ & $x_2 = 0$ & $x_2 = 1$ & $x_2 = 2$ & $f_2^*(s_2)$ & $x_2^*$ \\
\hline
0 & 0.48 & - & - & 0.48 & 0 \\
\hline
1 & 0.3 & 0.32 & - & 0.3 & 0 \\
\hline
2 & 0.18 & 0.2 & 0.16 & 0.16 & 2 \\
\hline
\end{tabular}

Pour la dernière étape ($n = 1$), nous avons $s_1 = 2$ (il y a 2 scientifiques à affecter).
Nous avons
\begin{align*}
 f_1(2,0) &= p_1(2,0).f_2*(2) = 0.4 \times 0.16 = 0.064, \\
 f_1(2,1) &= p_1(2,1).f_2*(1) = 0.2 \times 0.3 = 0.06, \\
 f_1(2,2) &= p_1(2,2).f_2*(0) = 0.15 \times 0.48 = 0.072.
\end{align*}
Aussi, $f_1*(2) = f_1(2,1)$ et $x_1* = 1$.
\begin{tabular}{|c|c|c|c|c|c|}
$s_1$ & $x_1 = 0$ & $x_1 = 1$ & $x_1 = 2$ & $f_1^*(s_1)$ & $x_1^*$ \\
\hline
0 & 0.064 & 0.06 & 0.072 & 0.06 & 1 \\
\end{tabular}
La politique optimale est donc :
$x_1* = 1$ (d’où $s_2 = 1$), $x_2* = 0$ (d’où $s_3 = 1$), $x_3^* = 1$.
 La probabilité d’échec est: $f_1^*(1) = 0.06$.
\end{example}

\section{Affectation de ressources}

La programmation dynamique est souvent utilisée pour résoudre des problèmes d’affectation de ressources.
C’était le cas dans notre dernier exemple, où il y avait une ressource à affecter (les scientifiques).
Que faire lorsqu’il y a plus d’une ressource?
Dans ce cas, chaque état est un vecteur dont chaque coordonnée $i$ représente la quantité de la ressource $i$ non encore affectée.

\begin{example}[problème Wyndor Glass]
Nous considérons deux types de produits (produit 1, produit 2) et trois usines (usine 1, usine 2, usine 3).
La ressource $i$ est la capacité de production pour l’usine $i$, exprimée en heures par semaine.
Nous connaissons le profit par lot (20 unités) de chaque produit, et chaque lot du produit 1 (2) est le résultat combiné de la production aux usines 1 et 3 (2 et 3).
Nous souhaitons déterminer le taux de production pour chaque produit (nombre de lots par semaine) de façon à maximiser le profit total.
Les données sont reproduites dans la Table~\ref{tab:wyndor_data}.
\begin{table}[htb]
\begin{tabular}{|c|c|c|c|}
\hline
& Produit 1 (temps de & Produit 2 (temps de & Capacité de \\
& production, h/lot) & production, h/lot) & production (h) \\
\hline
Usine 1 & 1 & 0 & 4 \\
\hline
Usine 2 & 0 & 2 & 12 \\
\hline
Usine 3 & 3 & 2 & 18 \\
\hline
Profit(\$)/lot & 3000 & 5000 & \\
\hline
\end{tabular}
\end{table}
L'étape $n$ sera le produit $n$, $n = 1,2$ ($N = 2$).
Nous avons comme état $s_n$ $(R_1,R_2,R_3)_n$, où $R_i$ est la quantité de la ressource $i$ (temps de production à l’usine $i$) non encore affectée.
La décision $x_n$ est le nombre de lots du produit $n$, et
\begin{align*}
 f_n(s_n,x_n) &= c_nx_n + f_{n+1}^*(s_{n+1})\ (f_3^*(s_3) = 0),\\
 f_2^*(s_2) &= max \lbrace 5x_2\,|\,2x_2 \leq12, 2x_2 \leq 18 - 3x_1, x_2 \geq 0 \rbrace \\
 f_1^*(s_1) &= max \lbrace 3x_1+f_2^*(s_2)\,|\,x_1 \leq 4, 3x_1 \leq 18, x_1 \geq 0 \rbrace
\end{align*}
De plus,
\[
 s_1= (4,12,18),\ s_2 = (4-x_1,12,18-3x_1).
\]

Afin de résoudre
 On cherche d’abord f_2*(s_2) et x_2*
 f_2*(s_2) = max{5x_2|2x_2\leq12, 2x_2\leq18-3x_1, x_2\geq0}
 2x_2\leq12 et 2x_2\leq18-3x_1 →x_2 \leq min{6,(18-3x_1)/2}
 f_2*(s_2) = max{5x_2|0\leqx_2\leq min{6,(18-3x_1)/2}}
 x_2* = min{6,(18-3x_1)/2}
 f_2*(s_2) = 5 min{6,(18-3x_1)/2)|3x_1\leq18}
 Calculons maintenant f_1*(s1) et x_1*
6. Programmation dynamique 28
Exemple 3 : résolution (suite)
 f_1*(s1) = max{3x_1+f_2*(s_2)|x_1\leq4,3x_1\leq18,x_1\geq0}
 x_1\leq4 →3x_1 \leq18
 f_1*(s1) = max{3x_1+5 min{6,(18-3x_1)/2)|0\leqx_1\leq4}
 min{6,(18-3x_1)/2)} = 6, si 0\leqx_1\leq2
= (18-3x_1)/2, si 2\leqx_1\leq4
 f_1*(s1) = max{f_1(s1,x_1)|0\leqx_1\leq4}
 f_1(s1,x_1) = 3x_1 + 30, si 0\leqx_1\leq2
= 45 – (9/2)x_1, si 2\leqx_1\leq4
 Le maximum est atteint en x_1* = 2 et f_1*(s1) = 36
6. Programmation dynamique 29
Exemple 3 : ræésolution (suite et fin)
 x_2* = min{6,(18-3x_1)/2} et x_1* = 2 →x_2* = 6
 La politique optimale est donc :
x_1* = 2, x_2* = 6 de valeur f_1*(s1) = 36
 En utilisant la même approche, on peut résoudre par
la programmation dynamique des modèles où :
 L’objectif ou les contraintes sont non linéaires
 Certaines variables sont entières
\end{example}

6. Programmation dynamique 30
Cas probabiliste
 Dans la programmation dynamique probabiliste, la
décision optimale à l’étape n dépend de la loi de
probabilité de l’état à l’étape n+1
 Ainsi, on passera de l’étape n à l’étape n+1, se
retrouvant ainsi dans l’état i, avec une probabilité pi
 Étant donné S états à l’étape n+1, on aura pi \geq 0 et
 La relation de récurrence dépend de cette loi de
probabilité et de l’objectif à optimiser
Σ
=
=
S
i
i p
1
1
6. Programmation dynamique 31
Exemple 4 : jeu de hasard
 Le jeu consiste à miser un nombre quelconque de
jetons
 Si on gagne, on gagne le nombre de jetons misés
 Si on perd, on perd le nombre de jetons misés
 Un statisticien croit pouvoir gagner chaque jeu avec
une probabilité égale à 2/3
 Ses collègues parient avec lui qu’en misant au départ
3 jetons, il aura moins de 5 jetons après 3 parties
 Combien de jetons miser à chacune des 3 parties?
6. Programmation dynamique 32
Exemple 4 : modèle
 Étape n = partie n, n = 1,2,3 (N = 3)
 État sn = nombre de jetons au début de la partie n
 Décision xn = nombre de jetons à parier à la partie n
 On veut maximiser la probabilité d’avoir au moins 5
jetons après 3 parties
 fn(sn,xn) : probabilité de terminer avec au moins 5
jetons, étant donné qu’on est à l’état sn à l’étape n,
qu’on mise xn jetons et qu’on effectue des décisions
optimales aux étapes n+1,…,N
 fn*(sn) = max{fn(sn,xn)| xn = 0,1,…,sn}
6. Programmation dynamique 33
Exemple 4 : modèle (suite)
 Étant donné qu’on est à l’état sn à l’étape n et qu’on
mise xn jetons, on peut :
 Perdre et se retrouver à l’état sn – xn avec une probabilité
1/3
 Gagner et se retrouver à l’état sn + xn avec une probabilité
2/3
 fn(sn,xn) = (1/3).fn+1*(sn-xn) + (2/3). fn+1*(sn+xn)
 f4*(s4) = 0, si s4 < 5
= 1, si s4 \geq 5
 Calculons d’abord f3*(s3) et x3*
6. Programmation dynamique 34
Exemple 4 : résolution (n = 3)
2/3
0
0
0
-
x3=1
2/3
2/3
0
-
-
x3=2
2/3
2/3
-
-
-
x3=3
2/3
-
-
-
-
x3=4
3 0 2/3 \geq2
2 0 0 \geq0
4 0 2/3 \geq1
1 0 0 \geq0
0 0 0 0
\geq 5 1 1 0
x3f * 3*(s3x ) 3s =0 3
6. Programmation dynamique 35
Exemple 4 : résolution (n = 2)
 Voyons maintenant comment on peut calculer les
valeurs f_2*(s_2) et x_2*, lorsque s_2 = 3
 Il est clair qu’on doit avoir x_2 \leq 3
 f_2(3,0) = 1/3.f3*(3) + 2/3.f3*(3) = 2/3
 f_2(3,1) = 1/3.f3*(2) + 2/3.f3*(4) = 4/9
 f_2(3,2) = 1/3.f3*(1) + 2/3.f3*(5) = 2/3
 f_2(3,3) = 1/3.f3*(0) + 2/3.f3*(6) = 2/3
 f_2*(3) = max{f_2(3,0),f_2(3,1),f_2(3,2),f_2(3,3)} = f_2(3,0)
 D’où x_2* = 0, 2 ou 3
 De la même façon, on calcule pour s_2 = 0,1,2,4,5
6. Programmation dynamique 36
Exemple 4 : résolution (n = 2)
8/9
4/9
4/9
0
-
x_2=1
2/3
2/3
4/9
-
-
x_2=2
2/3
2/3
-
-
-
x_2=3
2/3
-
-
-
-
x_2=4
3 2/3 2/3 0,2,3
2 0 4/9 1,2
4 2/3 8/9 1
1 0 0 \geq0
0 0 0 0
\geq 5 1 1 0
x_2f * 2*(s_2x ) 2s =0 2
6. Programmation dynamique 37
Exemple 4 : résolution (n = 1)
 Pour la première étape (n = 1), on a s1 = 3 (3 jetons
au départ affecter)
 f_1(3,0) = 1/3.f_2*(3) + 2/3.f_2*(3) = 2/3
 f_1(3,1) = 1/3.f_2*(2) + 2/3.f_2*(4) = 20/27
 f_1(3,2) = 1/3.f_2*(1) + 2/3.f_2*(5) = 2/3
 f_1(3,3) = 1/3.f_2*(0) + 2/3.f_2*(6) = 2/3
 f_1*(3) = max{f_1(3,0),f_1(3,1),f_1(3,2),f_1(3,3)} = f_1(3,1)
 D’où x_1* = 1
6. Programmation dynamique 38
Exemple 4 : résolution (suite et fin)
 La politique optimale est donc :
 x_1* = 1
 Si on gagne (s_2 = 4), x_2* = 1
 Si on gagne (s3 = 5), x3* = 0
 Si on perd (s3 = 3), x3* = 2 ou 3
 Si on perd (s_2 = 2), x_2* = 1 ou 2
 Si on gagne (s3 = 3 ou 4), x3* = 2 ou 3 (si x_2* = 1) ou
1\leqx3*\leq4 (si x_2* = 2)
 Si on perd (s3 = 1 ou 0), x3* \geq 0 (mais le pari est
perdu!)