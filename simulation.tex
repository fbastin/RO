\chapter{Simulation}

\section{Introduction}

Simuler un système stochastique consiste à imiter son comportement pour estimer sa performance
Un modèle de simulation est une représentation du système stochastique permettant de générer un grand nombre d'événements aléatoires et d'en tirer des observations statistiques.

\begin{example}[Jeu de hasard]
Chaque partie consiste à tirer une pièce de monnaie jusqu'à ce que la différence entre le nombre de faces et le nombre de piles soit égale à 3.
Chaque tirage coûte 1\$, et chaque partie jouée rapporte 8\$ au joueur.
Nous aurons par exemple les jeux
\begin{itemize}
\item
 FFF: gain de 8\$-3\$=5\$;
\item
 PFPPP: gain de 8\$-5\$=3\$;
\item
 PFFPFPFPPPP: perte de 8\$-11\$=3\$.
\end{itemize}
Nous nous demandons par conséquent s'il est intéressant de jouer.
Pour répondre à cette question, nous allons simuler le jeu.
Il y a deux façons de le faire:
\begin{itemize}
\item
nous pouvons jouer pendant un certain temps sans miser d'argent;
\item
nous pouvons simuler le jeu par ordinateur.
\end{itemize}
Néanmoins, simuler une seule partie ne nous aide pas à prendre une décision.
Pour cela, il faudrait voir ce qui se passe sur un grand nombre de parties et mesurer le gain moyen
(ou la perte moyenne).

Dans cet exemple, nous pouvons définir les éléments d'un modèle de simulation comme suit
\begin{description}
\item[système stochastique]: tirages successifs;
\item[horloge]: nombre de tirages;
\item[définition de l'état du système]: $N(t)$, le nombre de faces moins le nombre de piles après $t$ tirages;
\item[événements modifiant l'état du système]: tirage de pile ou de face;
\item[méthode de génération d'événements]: génération d'un nombre aléatoire uniforme;
\item[formule de changement d'état]: $N(t+1) = N(t) + 1$, si $F$ est tirée; $N(t) - 1$, si $P$ est tirée;
\item[performance]: $8 - t$, lorsque $N(t)$ atteint +3 ou -3.
\end{description}
\end{example}

\section{Files d'attente}

\subsection{Concepts de base}

Un système de file d'attente consiste d'un ou de plusieurs serveurs
qui fournissent un service d'une certaine nature à des clients qui se
présentent, et qui sont issus d'une population donnée.
Les clients qui arrivent et trouvent tous les serveurs occupés
rejoindront (généralement) une ou plusieurs files (ou lignes) devant
les serveurs, d'où la qualification de file d'attente.
\index{file!d'attente}.
Quelques exemples de files d'attente sont repris dans la
Table~\ref{tab:queueing}.
\begin{table}[htb]
\begin{center}
\begin{tabular}{lll}
{\bf système} & {\bf Serveurs} & {\bf Clients} \\
\hline
Banque & Guichets & Clients \\
Hôpital & Docteurs, infirmiéres, lits &  Patients \\
Système informatique & Unités centrales, dispositifs I/O & Travaux \\
Manufacture & Machines, travailleurs & Composants \\
Centre d'appel & Serveurs & Appels entrants \\
\hline
\end{tabular}
\end{center}
\caption{Exemples de files d'attente}
\label{tab:queueing}
\end{table}
Le système de file d'attente est caractérisé par trois composants:
\begin{enumerate}
\item le processus d'arrivée;
\item le mécanisme de service;
\item la discipline de file.
\end{enumerate}

Spécifier le processus d'arrivée pour un système de file d'attente
revient à décrire comment les clients arrivent dans le système. Soit
$A_i$ le temps d'inter-arrivées entre les arrivées des $(i-1)^e$ et
$i^e$ clients. Si $A_1$, $A_2,\ldots$, sont supposés être des
variables i.i.d., nous dénoterons le temps d'interarrivée moyen (ou
espéré) par $E[A]$ et appellerons $\lambda = 1/E[A]$ le taux d'arrivée
des clients.

Le mécanisme de service pour un système de file d'attente est articulé
en spécifiant le nombre de serveurs (dénoté par $s$), où chaque
serveur a sa propre file ou au contraire une seule file alimente tous
les serveurs, et la distribution de probabilité des temps de service
des clients.
Soit $S_i$ le temps de service du $i^e$ client.
Si $S_1$, $S_2,\ldots$, sont des variables aléatoires i.i.d., nous dénoterons
le temps de service moyen d'un client par $E[S]$ et appelleront
$\omega = 1/E[S]$ le taux de service d'un serveur.

Nous parlerons de situation transitoire lorsque l'état du système dépend grandement de la situation initiale et du temps écoulé, et de situation d'équilibre lorsque l'état du système peut
être considéré indépendant de la situation initiale et du temps écoulé.
En situation d'équilibre, nous avons la formule de Little:
\[
 L=\lambda W,
\]
où
$L$ est le nombre moyen de clients dans le système, $\lambda$, le taux moyen d'arrivée des nouveaux clients, et $W$ le temps moyen dans le système.

\subsection{Modèle $M/M/1$}

Il s'agit du modèle de file d'attente le plus courant, avec les caractéristiques suivantes:
\begin{itemize}
\item
file d'attente: nombre infini de clients;
\item
stratégie de service: premier arrivé, premier servi (FIFO);
\item
un seul serveur;
\item
les taux d'arrivée et de service obéissent à des lois de Poisson.
De manière équivalente, le temps entre l'arrivée de deux clients successifs et le temps de service obéissent à des lois exponentielles: on parle de processus Markoviens.
\end{itemize}

La notation générale pour les modèles de file d'attente est $X/Y/s$, où $X$ représente la loi du temps d'interarrivée, $Y$ est la loi du temps de service, et $s$ donne le nombre de serveurs.

\begin{example}[File d'attente $M/M/1$]
En situation d'équilibre, plusieurs résultats analytiques (obtenus par analyse du modèle mathématique) existent.
Soit $\lambda$ le taux moyen d'arrivée et $\mu$ le taux moyen de service.
Supposons que $\lambda < \mu$
Le nombre moyen de clients dans le système vaut
\[
L = \frac{\lambda}{\mu-\lambda}.
\]
Le temps moyen d'attente dans le système vaut quant à lui
\[
W = \frac{1}{\mu-\lambda}.
\]
Il est aisé de vérifier ces résultats par simulation.
Dans le cas présent, les éléments de la simulation sont:
\begin{description}
\item[système stochastique]: file d'attente $M/M/1$;
\item[horloge] : temps écoulé;
\item[définition de l'état du système]: $N(t)$, le nombre de clients dans le système au temps $t$;
\item[événements modifiant l'état du système]: arrivée ou fin de service d'un client;
\item[formule de changement d'état] : $N(t+1) = N(t) + 1$, si arrivée; $N(t) - 1$, si fin de service.
\end{description}
Nous allons voir deux méthodes pour étudier l'evolution du système dans le temps:
\begin{itemize}
\item
 par intervalles de temps fixe;
\item
 par génération d'événement.
\end{itemize}
Nous supposerons que les valeurs des paramètres de notre système sont $\lambda$ = 3 clients par heure, et $\mu$ = 5 clients par heure.

Dans le cas d'intervalles de temps fixe, l'idée est
\begin{enumerate}
\item
faire écouler le temps d'un petit intervalle $\Delta t$;
\item
mettre à jour le système en déterminant les événements qui ont pu se produire durant l'intervalle
$\Delta t$, et recueillir l'information sur la performance du système;
\item
retourner en 1.
\end{enumerate}
Ici, les événements sont soit des arrivées, soit des départs (fins de service).
Si $\Delta t$ est suffisamment petit, nous pouvons considérer qu'il ne se produira qu'un seul événement (arrivée ou départ) durant cet intervalle de temps.

Prenons $\Delta t$ égal à 0.1 heure (6 minutes).
La probabilité qu'il y ait une arrivée durant cet intervalle de temps vaut:
\[
P_A = 1 - e^{-\lambda \Delta t} = 1-e^{-3/10} \approx 0.259.
\]
La probabilité qu'il y ait un départ durant cet intervalle de temps est:
\[
P_D = 1 - e^{-\mu \Delta t} = 1-e^{-5/10} \approx 0.393.
\]
Pour générer un événement, nous pouvons simplement tirer deux nombres aléatoires selon une loi $U[0,1]$.
Si le premier nombre est strictement plus petit que 0.259, nous aurons une arrivée.
Si le deuxième nombre est strictement inférieur à 0.393, nous considérerons un départ (si un client était en cours de service).
Ceci pourrait par exemple donner les chiffres de la Table~\ref{tab:simultime}.

\begin{table}[htbp]
\begin{center}
\begin{tabular}{|c|c|c|c|c|c|}
\hline
$t$ (min) & $N(t)$ & Nombre 1 & Arrivée & Nombre 2 & Départ \\
\hline
0 & 0 & & & & \\
\hline
6 & 1 & 0.096 & Oui & - & \\
\hline
12 & 1 & 0.569 & Non & 0.665 & Non \\
\hline
18 & 1 & 0.764 & Non & 0.842 & Non \\
\hline
24 & 1 & 0.492 & Non & 0.224 & Oui \\
\hline
30 & 0 & 0.950 & Non & - & \\
\hline
36 & 0 & 0.610 & Non & - & \\
\hline
42 & 1 & 0.145 & Oui & - & \\
\hline
48 & 1 & 0.484 & Non & 0.552 & Non \\
\hline
54 & 1 & 0.350 & Non & 0.590 & Non \\
\hline
60 & 0 & 0.430 & Non & 0.041 & Oui \\
\hline
\end{tabular}
\caption{Simulation par intervalles de temps fixe.}
\label{tab:simultime}
\end{center}
\end{table}
D'après cet exemple, il est possible d'estimer les performances du système.
Que se passe-t-il si nous voulons mesurer $W$, le temps moyen passé dans le système?
Nous avons deux clients qui sont entrés dans le système et chacun y est resté 18 minutes ou 0.3 heures, aussi peut-on estimer $W$ par 0.3.
Pourtant, la vraie valeur est $W = 1/(\mu-\lambda) = 0.5$.
Il faudrait un \'echantillon beaucoup plus grand, et ce d'autant plus que nous devrions simuler le système en état d'équilibre.

La génération d'événement peut se résumer comme suit:
\begin{enumerate}
\item
faire écouler le temps jusqu'au prochain événement;
\item
mettre à jour le système en fonction de l'événement qui vient de se produire et générer aléatoirement le temps jusqu'au prochain événement; recueillir l'information sur la performance du système;
\item
retourner à 1.
\end{enumerate}
Nous pourrions ainsi obtenir la suite d'événements reprise dans la Table~\ref{tab:simulevents}.

\begin{table}[htbp]
\begin{center}
\begin{tabular}{|c|c|c|c|c|c|c|}
\hline
$t$ (min) & $N(T)$ & Temps & Temps &  Prochaine & Prochain & Prochain \\
& & d'interarrivée & de service & arrivée & départ & événement \\
\hline
0 & 0 & 2.019 & - & 2.019 & - & Arrivée \\
\hline
2.019 & 1 & 16.833 & 13.123 & 18.852 & 15.142 & Départ \\
\hline
15.142 & 0 & - & - & 18.852 & - & Arrivée \\
\hline
18.852 & 1 & 28.878 & 22.142 & 47.730 & 40.994 & Départ \\
\hline
40.994 & 0 & - & - & 47.730 & - & Arrivée \\
\hline
47.730 & 1 & & & & & \\
\hline
\end{tabular}
\caption{Simulation par événements.}
\label{tab:simulevents}
\end{center}
\end{table}

Si nous considérons une simulation comportant l'arrivée de 10000 clients, les résultats montrent que:
\begin{itemize}
\item
 le nombre moyens de clients dans le système est $L \approx 1.5$;
\item
 le temps moyen dans le système est $W \approx 0.5$.
\end{itemize}

Remarque: des résultats analytiques existent pour des modèles simples (comme $M/M/1$), mais pas pour des files d'attente plus complexes.
\end{example}

\section{Simulation à événements discrets}

L'approche s'est imposée dans nombre de problèmes de simulations.
Le principe général est résumé dans l'ordinogramme de la Figure~\ref{fig:sim_flowchart}.

\begin{figure}[htb]

\tikzstyle{decision} = [diamond, draw, fill=blue!20,
    text width=2.5cm, text badly centered, node distance=2.5cm, inner sep=0pt]
\tikzstyle{block} = [rectangle, draw, fill=blue!20,
    text width=2.5cm, text centered, rounded corners, minimum height=4em]
\tikzstyle{line} = [draw, very thick, color=black!50, -latex']
\tikzstyle{cloud} = [draw, ellipse,fill=yellow!20, node distance=2.5cm,
    minimum height=2em]

\begin{center}
\begin{tikzpicture}[scale=2, node distance = 3cm, auto]
    % Place nodes
    \node [block] (init) {Initialisation};
    \node [text width=2.2cm, cloud, left of=init, node distance=4.0cm] (clock) {Horloge de simulation mise à 0};
    \node [text width=2.5cm, cloud, right of=init, node distance=4.0cm] (system) {Planification du ou des premiers(s) événement(s)};
    \node [text width=2.1cm, block, below of=init, node distance=3.0cm] (event) {Gestion de l'événement actuel};
    \node [text width=2.1cm, block, below of=event, node distance=3.0cm] (schedule) {Planification du ou des événements en découlant.};
    \node [block, left of=schedule, node distance=5.0cm] (update) {Dépilement du prochain événement et avancement de l'horloge de simulation};
    \node [text width=2.1cm, decision, below of=schedule, node distance=4.0cm] (decide) {Liste d'événements vide?};
    \node [block, below of=decide, node distance=3.5cm] (stop) {Arrêt};
    % Draw edges
    \path [line] (init) -- (event);
    \path [line] (event) -- (schedule);
    \path [line] (schedule) -- (decide);
    \path [line] (decide) -| node [near start, color=black] {non} (update);
    \path [line] (update) |- (event);
    \path [line] (decide) -- node [, color=black] {oui}(stop);
    \path [line,dashed] (clock) -- (init);
    \path [line,dashed] (system) -- (init);
\end{tikzpicture}
\caption{Schéma d'une simulation à événements discrets.}
\label{fig:sim_flowchart}
\end{center}
\end{figure}

\begin{small}
\section{Notes}

Ce chapitre se base essentiellement sur les notes de cours de Bernard Gendron, 2007.
Pour plus de détails sur la simulation à événements discrets, nous renvoyons le lecteur aux notes du cours IFT3245, du même auteur: .

\end{small}
